\section{What did you learn?}

The length and depth of this project allowed us to learn a great many things,
consisting of new technologies and new work methodologies. Key to this is the
agile workflow, which was required for this project. We all learnt how using
agile methods can be better for quick prototyping and helping to focus our
project early through frequent refinement. This was also necessary when our
requirements changed throughout the project, which would've meant extensive
rework if we weren't using agile.

Some of the technologies new to us was the Unity game engine, which was required
to use for the racing simulation. Only one group member had previously used this
before, which helped a great deal with the initial work but other group members
had to learn the framework and its intricacies. 

\begin{itemize}
    \item
        How \textit{fun} asynchronous javascript is
\end{itemize}

\section{What would we do differently? TODO}

A problem we encountered with deploying the Unity side of the game was that the
compiled web build had to be made with the Unity editor, and so was included in
the git repository as this is what was transferred to the VM our project ran on.
This web build was very large, and resulted in significantly long build times,
as well as causing us problems with disk space on the VM, since git would keep
all previous versions of it as part of its history. Towards the end of the
project, we found that there were ways of removing this build from previous
commits, which would help with our disk space problems. We also managed to find
a command line version of Unity that could compile the web build on the VM,
although this would have to run on Windows, whether through another virtual
machine or just having the one we used as Windows instead of Linux. We didn't
change this around as we had almost finished, and it could potentially break our
project at an important time if we tried to alter this too much. However, this
would be something we would strongly consider doing were we to do it again,
since it would solve a vast amount of the annoyances we encountered above.

\begin{itemize}
    \item
        Try and stick better to agile?
    \item
        Try and get more frequent feedback (smaller sprints?)
    \item
        Testing
    \item
        Do more feedback sessions with users, i.e. do a few practice runs with
        groups of students to see if it's working well
\end{itemize}

\section{Future extensions}
% There can probably be a lot for this... list the key features we would
% want to add, or the ones we were planning on adding before we ran out of time

Whilst we got the key features completed on time, there are a few things we were 
thinking about doing towards the end of the project that we could do if we had
more time to work on it. G-Research have mentioned that this project could
potentially be continued by another university, so these extensions could
eventually form a specification for them to work on.

Below is a list of the extensions ideas we have and the reasons for them:

\begin{itemize}
    \item
        \textbf{More Extensive API} - Allowing more control over the car allows
        for more complex and detailed AI scripts, and could help find the best
        candidates who can use all the features. Examples of this include 
    \item
        \textbf{Better car physics} - Our simulated car physics could do with
        some improvements - for example when moving backwards the car has a
        tendency to flip over for no reason. The brake system also doesn't work
        that well with the road, meaning a negative throttle is more effective
        at stopping than braking is.
    \item
        \textbf{Cleaner setup} - Currently our deployment system consists of
        downloading the Github repository, and making sure to install the
        prerequisites (MongoDB and Node), then running the web server. This
        could be packaged up into an installer that bundles the compiled code
        and the prerequisites together for easy installation.
    \item
        \textbf{Better security} - We don't have much security for our project,
        mainly because it was only required to run at a careers fair in a closed
        network, which wouldn't be vulnerable anyway. However, those with
        knowledge of the source code could potentially use it to their advantage
        and modify the rules of the simulation, allowing them to perform better.
    \item
        \textbf{Better documentation} - More extensive documentation along with
        some example scripts on how to use the API would definitely aid future
        users with how best to go about writing a racing AI.
    \item
        \textbf{Improving other scripts} - this was originally part of the
        original specification, which wanted us to allow new players to see
        existing code and tweak it for better performance. We decided against
        this after we found this would be used at careers fairs to test
        candidates, as this would be unfair to the original authors of the
        scripts, and may lead to unfair perfomance comparisons.
        However, if the project were to be converted to more of a market than a
        performance tool, then this could then be integrated.
\end{itemize}



\section{Final words TODO}
