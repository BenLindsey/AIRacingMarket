\section{Team Development}

The length and depth of this project allowed us to learn a great many things, consisting of new technologies and new work methodologies. Key to this is the agile workflow, which was required for this project. We all learnt how using Agile methods can be better for quick prototyping and helping to focus our project early through frequent refinement. This was also necessary when our requirements changed throughout the project, which would've meant extensive rework if we weren't being agile.

One of the technologies new to us was the Unity game engine, which we used for the racing simulation. Only one group member had previously used this before, which helped a great deal with the initial work, however other group members had to quickly learn the framework and its intricacies. 

We also had to learn Node.js and the Express framework, which made us deal with a lot of asynchronous JavaScript callbacks, and the model-view-controller design pattern. Whilst we have encountered some of this before, we now feel more comfortable in this environment, and also how it links together with Unity, since the website had to send commands to our Unity game and vice versa.

Working with MongoDB instead of an SQL database is also different, since we no longer have relational databases and store most things in the same collection; we managed to learn a bit more about the structure of NoSQL databases and how you have to use them in a certain way that differs from what we are used to.

\section{Project Reflections/Improvements}
% Some of this feels more like evaluation than reflecting back, although I
% guess it could be both - perhaps consider shuffling things around?

We feel that a lot of our initial project work went well, with a prototype quickly forming and features being organised quickly. Looking back now, it is easy to judge the tasks that were more important that we could've worked on from the start, especially since G-Research have pointed out more of what they wanted as the project went on. Most of what we did went well and worked for us, so we wouldn't want to change how we worked drastically, but there were a few key issues that if solved early would've resulted in fewer problems for us later.

One of these problems was with deploying the Unity side of the game. The compiled web build had to be made with the Unity editor, and so was included in the git repository as this is what was transferred to the VM our project ran on.  This web build was very large, and resulted in significantly long build times, as well as causing us problems with disk space on the VM (since git would keep all previous versions of it as part of its history). Towards the end of the project, we found that there were ways of removing this build from previous commits, which would help with our disk space problems. We also managed to find a command line version of Unity that could compile the web build on the VM, although this would have to run on Windows, whether through another virtual machine or just having the one we used as Windows instead of Linux. We didn't change this around as we had almost finished, and it could potentially break our project at an important time if we tried to alter this too much. However, this would be something we would strongly consider doing were we to do it again, since it would solve a vast amount of the annoyances we encountered above.

Another problem was using the MongoDB database, although mostly at our own fault. We made some initial collections that we thought would be separate, but found later that we needed to `join' them. However, this cannot be done easily with NoSQL, and eventually we created a new collection to house all the data. We felt slightly uncomfortable using something that didn't have relational data, and never really got used to the paradigm shift, so perhaps using a relational model would've sped up our working and avoided these problems. At the same time, it is good to know about NoSQL databases, and we would not have learnt about it during this project otherwise, so we would need to consider which tradeoffs are worth it.

Whilst we tried to adhere to Agile methods when working on this project, we quickly fell into the trap of not rotating our team members through the various project technologies. This was in part due to laziness, but also because we often wouldn't work as a complete group. This meant that members who were experienced with specific parts of the project could not support others that had less experience - this was to some extent  inescapable.

%Whilst we tried to adhere to Agile methods when working on this project, we found that we didn't do everything we could have done. Scrums weren't consistent, mainly because we worked together in the same place and it wasn't as necessary to always stop working and discuss working - instead they happened only when absolutely necessary (i.e. when designing a new feature). We also didn't rotate our team members a lot. Using more agile methods would've taught us a lot more about this way of working, as well as potentially improving the work we did.

Our sprints were two weeks long, which seemed to work well due to the other work we had to complete in the time frame (coursework and such), but more feedback more frequently is always better, so perhaps shortening this would've improved our work. Unfortunately our client could only commit to four meetings, and clearly it was important that he attended our Sprint Reviews to help steer our project.

\section{Future Extensions}

Whilst we got the key features completed on time, there are a few things we were thinking about doing towards the end of the project that we could have done if given more time. G-Research have mentioned that this project could potentially be continued by another university, so these extensions could eventually form a specification for them to work on; G-Research has pitched this same project specification at both Cambridge and Oxford. Below is a list of the extensions ideas we have and the reasons for them:

\begin{itemize}
    \item
        \textbf{More Extensive API} - Allowing more control over the car and the world around it allows for more complex and detailed AI scripts, and could help find the candidates who make best use of the features.  Examples of this include more environmental (such as use of the track gradient) and situational awareness (such as whether opponents are using powerups), as well as extra powerups.
    \item
        \textbf{Extra Languages} - Scripts can currently be written in JavaScript or CoffeeScript. This could be extended to other languages, allowing users to script in the language they are most comfortable with.
    \item
        \textbf{Better Car Physics} - Our simulated car physics could do with some improvements - for example when moving backwards the car has a tendency to flip over for no reason. The brake system also doesn't work that well with the road, meaning a negative throttle is more effective at stopping than braking is.
    \item
        \textbf{Cleaner Deployment} - Currently our deployment system consists of downloading the Github repository, and making sure to install the prerequisites (MongoDB and Node.js), then running the web server. This could be packaged up into an installer that bundles the compiled code and the prerequisites together for easy installation.
    \item
        \textbf{Better Security} - We don't have much security for our project, mainly because it was only required to run at a careers fair in a closed network, reducing vulnerability. However, those with knowledge of the source code could potentially use it to their advantage and modify the rules of the simulation, allowing them to perform better.
    \item
        \textbf{Better Documentation} - More extensive documentation along with some example scripts on how to use the API would definitely aid future users with how best to go about writing a racing AI. You can never have too much documentation!
    \item
        \textbf{Improving Other Scripts} - This was part of the original specification, which wanted us to allow new players to see existing code and tweak it for better performance. We decided against this after we found this would be used at careers fairs to test candidates, as this would be unfair to the original authors of the scripts, and may lead to unfair performance comparisons. However, if the project were to be converted to more of a market than a performance tool, then this could then be integrated.
\end{itemize}
