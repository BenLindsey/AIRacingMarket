You're in a bright, hot, crowded room and a fellow student has just shoved past you for the fifth time today. Corporate banners stretch from floor to ceiling while the largest pyramid of free mugs you've ever seen wobbles slightly. This is the third careers fair you've visited so far and you're simply one branded pen away from madness.

G-Research's job is to grab your attention, get to know you a little, and ultimately harvest your email address. Now this is no easy task, the competition this year is fierce and the lollipops\footnote{G-Research's stash of flavourless lollipops remained largely intact after their last careers fair. As a reward for finishing our project, we were offered a box. We declined.} aren't quite drawing in the expected crowds.

\project is our solution to this problem. It's a competitive platform that allows students to easily script racing AIs, then simulate these in 3D. Tournaments run automatically, constantly showing a clash of high speed cars and collisions to attract passing students. Meanwhile, leaderboards track the best contenders allowing G-Research to offer prizes and discover the best prospective employees. 

Although our project was primarily designed to operate in a careers fair, spread across several laptops, the website can also be accessed from home. This encourages particularly enthusiastic students to compete after the careers fair, drumming up interest and brand recognition even after the main recruitment push has finished.
%This project is co-supervised by G-Research, a financial services company that evaluates, forecasts and simulates investment ideas across multiple financial asset classes. They produce software and services across a range of markets, allowing the deployment of these investment ideas and portfolio risk management.