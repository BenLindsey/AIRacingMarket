\section{System Architecture}

Our first step, after reading the description of our project, was to decide what kind of application or system would be best suited to tackle our requirements. The first thing that we gleaned from the specification was that some form of persistent storage would be required. For example, the user submitted scripts must be stored so that they can later be raced against new scripts and data used to rank the scripts must be recorded so that a leaderboard can be shown. It was also mentioned in our project's specification that we should use the Unity game engine for the physics simulation and visualisation of our vehicle races. These two requirements gave us a solid starting point for designing our system. Our first thought was to develop a stand alone Unity application to run on any desktop machine. This option would have required one machine to act as a server and manage the centralised data while the other machines communicate with it using networking code inside of Unity. It would also have meant we'd need to build a full user interface for uploading, editing and managing scripts inside of Unity. The more we thought about this the more we realised that these tasks would be a massive undertaking and highly difficult to achieve in the time frame, due to our group's limited knowledge of Unity.

Once we began thinking about alternatives an obvious solution sprang to mind. A huge benefit of Unity is that it is highly portable and can be run on many different platforms. One of these platforms is the Unity Web Player, a browser plug-in that allows Unity applications to be run inside all modern web browsers. If we used the web player we would could pull all of the things that are difficult to do in Unity outside of the Unity application and into the web browser. Thus turning the task of creating an easy to use interface for uploading scripts into a web design problem, a domain that is a lot better suited to solve user interface tasks such as this. The Unity application is then only needed for running and visualising the races. Unity also provides features for communication with the web browser, in both directions, when running in the web player, making passing data to the Unity application at runtime easy. This design also reduces the complexity of the networking problems as now one machine can run a web-server and database, using standard web technologies, while client machines simply run a web browser, containing the Unity Web Player, and communicate with the server using standard web protocols.

\subsection{Component Diagram}

\tikzset{%
  block/.style    = {draw, thick, rectangle, minimum height = 3em, rounded corners,
    minimum width = 3.5em, align=center},
  sum/.style      = {draw, circle, node distance = 2cm}, % Adder
  %input/.style    = {coordinate},
  db/.style  = {draw, cylinder, shape border rotate=90, aspect=0.25},
  interface/.style    = {draw, thick, rectangle,
    minimum width = 4em, align=center},
}

\begin{center}
\begin{tikzpicture}[auto, thick, node distance=2cm, >=triangle 45]
	% Blocks of FRONT END:
	\draw
		node at (1, -1.7) [block] (unity) {Game \\ Objects}
		node [block, right of=unity, right=-0.25cm] (webplayer) {Unity \\ Webplayer}
		node [block, right of=webplayer, right=-0cm] (unityscript) {Unity \\ Script}
		node [block, right of=unityscript, right=-0.35cm] (ai) {AI\\Script}
		node [block, right of=ai, right=-0.5cm] (browser) {Browser}
		%node [sum, right of=input1] (suma1) {Unity}
	;
    	% Joining FRONT END:
	\draw[->] (browser) -- node {} (ai);
	\draw[->] (ai) -- node {} (unityscript);
	\draw[->] (unityscript) -- node {} (webplayer);
	\draw[->] (webplayer) -- node {} (unity);
	\draw[-] (webplayer) |- node[right=3.5cm, above] {\small Request Commands } ($(browser.north) + (0, 0.5)$);
	\draw[->] ($(browser.north) + (0, 0.5)$) -- node {} (browser);

	% Blocks of MIDDLE
	\draw
		node at (5.5,-4) [block, name=rest] {REST API}
	;

        % Blocks of BACK END
	\draw
	        node [block, below of=unity, below=2.3cm, name=express] {Express\\Router}
	        node [block, right of=express, right=0cm, name=passport] {Passport \\ Auth.}
	        node [db, below of=passport, below=0cm, name=mongo] {MongoDB}
	        node [interface, above of=mongo, above=-1cm, name=monk] {Monk Adapter}
	        node [block, right of=passport, right=0cm, name=routes] {Routes}
	        node [block, right of=routes, right=0cm, name=views] {Views}
	;
	% Joining BACK END
	\draw[->] (express) -- node {} (passport);
	\draw[->] (rest) -| node {} (express);
	\draw[->] ($(browser.south) + (-0.2, 0)$) |- node[left=2.25cm, below] {\small Request Page} (rest);
	\draw[-, dotted] (passport) -- node[below right] {} (monk);
	\draw[-] (monk) -- node {} (mongo);
	\draw[->] (passport) -- node {} (routes);
	\draw[->] (routes) -- node {} (views);
	\draw[-, dotted] (routes) |- node {} (monk);
	\draw[->] (views) -| node {} ($(browser.south) + (0.2, 0)$);

	% Boxing
	%\draw [color=gray,thick, dotted] ($(passport.north west)+(-0.2,0.2)$) rectangle ($(monk.south east)+(0.2,-0.2)$);
	\draw [color=gray,thick](-0.5,-3) rectangle (12.55,0);
	\node at (-0.5,0) [above=5mm, right=0mm] {\textsc{Front-End}};
	\draw [color=gray,thick](-0.5,-10.5) rectangle (12.55,-5);
	\node at (-0.5,-10.5) [below=5mm, right=0mm] {\textsc{Back-End}};
\end{tikzpicture}
\end{center}

\section{Back-end}
The architecture of our server revolves around providing a website to upload, manage and score AI scripts. We provide these services via a simple REST API, implemented with Node.js\cite{whynode} and its supporting libraries:
	\begin{itemize}
	        \item \textbf{Express}\cite{express} is a light-weight web application framework that supports the MVC design pattern using routes (controllers) and views.  
		\item \textbf{Passport}\cite{passport} handles authentication for our website. We currently use the 'local' package which provides account registration and login via our Mongo database. However, in the future we could easily slot in Facebook or Google packages to provide alternative login methods.	
		\item \textbf{Monk}\cite{monk} provides a simple adapter to access our Mongo database in JavaScript. It uses an asynchronous tactic of searching by a filter object (e.g. to find a user called Steve, you'd filter with a new user object with name Steve) combined with a closure to be invoked on the found objects. 
\begin{figure}[H]
\centering
\begin{lstlisting}[language=JavaScript]
router.get('/script/:name', function(req, res) {
    db.get('scriptcollection').findOne({scriptName:req.params.name},  
        function(e, doc) { res.send(doc.script, 200); }
    );
}
\end{lstlisting}
\caption{Example use of Express (router.) \& Monk (db.) to GET a script.}
\label{fig:getscript}
\end{figure}
		
		\item \textbf{Forever} solves one of the main limitations of Node.js: that in the event of an exception the website will crash and shut down. Forever is a simple way to keep a script, and thus the website it controls, operating continuously. In the event of an uncaught exception, server restart or otherwise fatal condition Forever will restart the website immediately. Forever also maintains a log of such circumstances to help track down the cause.
	\end{itemize}

\subsection{REST API}
A Representational State Transfer (REST) design provides us with a stateless, cacheable interface for accessing our server. It unifies the various responsibilities of our back-end, such as providing webpages and updating script scores, behind a single vocabulary of HTTP requests:
\begin{figure}[H]
\centering
\begin{tabular}{| l | l | l | l |}\hline
Route & Request &  Response & Explanation\\\hline\hline
/script & GET & View & Show the create new script page\\\hline
/script & POST & None & Store a new script to the server\\\hline
/script/DrEvil & GET & String & Get the code for script DrEvil [see Figure \ref{fig:getscript}]\\\hline
/edit/DrEvil & GET & View & Show the live edit script page\\\hline
/tournament & GET & View & Show the next tournament match\\\hline
/tournament & POST & View & Store the match result and show the next \\\hline
\end{tabular}
\caption{Example interactions specified by our REST API.}
\label{fig:api}
\end{figure}

Figure \ref{fig:api} demonstrates the loosely-coupled nature of REST APIs. There are no inherant requirements placed on either the front or back end implementations. Our website could be accessed by a mobile or desktop device, while our server could be a local laptop, a VM cluster or a teapot\cite{rfc}. This allows us to update, or even completely scrap, either side of our architecture without having to fix the other.

\subsection{MongoDB}
For our database persistence we use an open source, document based solution - MongoDB\cite{mongo}. Mongo provides fast, scalable data storage with rich, object based queriers.

Our main motivation for using MongoDB was the flexibility its schema-less design offered\cite{whymongo}. Rapidly prototyping new ideas and concepts is much more productive when the updating and deployment of database schemas can be ignored. During such development data entries can frequently gain and lose attributes, which requires no interaction with the database when Mongo is used. Furthermore, by using the Monk adapter we can implement our full back-end in a single language, JavaScript, without complicated SQL statements.

\begin{figure}[H]
\centering
\begin{tikzpicture}[node distance=7 em]
\node [entity] (person) {User};
\node [relationship] (has) [right of=person] {Owns};
\node [entity] (script) [right of=has] {Script};
\node [attribute] (email) [above of=person] {\key{Email}} edge (person);
\node [attribute] (pid) [below of=person] {Year?} edge (person);
\node [attribute] (pid) [below left of=person, above=0cm] {Degree?} edge (person);
\node [attribute] (pid) [left of=person, left=-0.6cm] {University?} edge (person);
\node [attribute] (pid) [above left of=person, above=-0.5cm] {Hash} edge (person);
\node [attribute] (pid) [above of=script] {\key{Name}} edge (script);
\node [attribute] (pid) [below of=script] {Code} edge (script);
\node [attribute] (pid) [above right of=script] {Language} edge (script);
\node [attribute] (pid) [below right of=script] {Rating} edge (script);
%Edge
\path[every node/.style={font=\sffamily\small}] (person) edge node [above] {0..N} (has);
\path[every node/.style={font=\sffamily\small}] (script) edge node [above] {1..1} (has);
\end{tikzpicture}
\caption{ER diagram for our data persistence.}
\label{fig:ER}
\end{figure}

From Figure \ref{fig:ER} you can see another justification for our use of NoSQL technologies: the relationship is very simple. Document based technologies such as Mongo can struggle to provide simple modelling of complicated relationships, in these situations SQL based implementations typically offer a better solution. 

\section{Browser Front-End}

For the browser front-end, i.e. website, we used three technologies : Jade, Bootstrap and the Ace editor.


\subsection{Jade}
As explained above, we chose to use the Node.js platform, and we next had to chose a template engine for it's web template system. We chose to use Node's default engine, Jade\cite{jade}, which is written in JavaScript and specifically a templating language for HTML. It simplifies writing web pages as it is less verbose than HTML, and allows template inheritance (demonstrated below) alongside other dynamic constructs such as conditionals and loops.
\vspace{-2mm}
\begin{figure}[H]
\centering
\includegraphics[scale=0.25]{jadeexample.png}
\caption{Jade template inheritance is simple and powerful.}
\end{figure}

\noindent Compared to other engines such as Swig and Hamljs, which have similar feature sets, Jade has had reported sluggish benchmarking performance \cite{benchmarks} (obviously some benchmarks can't be treated as gospel). However, performance was not a concern for us; templating engines are very rarely a system bottleneck, and the decision to use Jade was founded upon it's usability and readability.

\subsection{Bootstrap}
Given our limited experience using Jade/HTML and CSS, alongside our negligible artistic talent, we opted to use Twitter's Bootstrap to design our website. Bootstrap makes web design a breeze by providing the HTML and CSS templates for well-designed web components, including forms, buttons, navigation bars, JavaScript extensions and more. In particular, we used the Sandstone\cite{sandstone} template from Bootswatch (released under MIT license), demonstrated in the figure below. Although this may not give users with a particularly unique experience, it means providing users with a familiar experience that would allow them to utilise preconceived assumptions on how the website can be navigated - avoiding the need to explain the website explicitly.

\begin{figure}[H]
\centering
\includegraphics[width=0.5\textwidth]{sandstonetheme.png}
\caption{Bootswatch Sandstone theme sample.}
\end{figure}

\subsection{Ace}
Apart from the most cavalier advocates of text editors, programmers shudder at the idea of coding in a plain text area. To simplify the process of reading, writing and submitting scripts, we chose to use a web-based code editor. There are few open-source editors that can rival the performance, features and simplicity of Ace (a standalone editor written in JavaScript). Ace\cite{ace} is actively developed for use in the online Cloud9 IDE, yet can be embedded in any web page (see picture below) given its BSD license.

\begin{figure}[H]
\centering
\includegraphics[width=0.75\textwidth]{aceeditorexample.png}
\caption{Ace editor containing code needed to embed in a web page.}
\end{figure}

\noindent By using Ace, users have JavaScript syntax highlighting, syntax error checking, searching/replace, automatic indentation etc. Not only does this improve the productivity of a user, which is important given the time pressures which may be faced if a queue forms at the G-Research careers fair stand; but it provides immediate feedback for those unfamiliar with JavaScript syntax, which may be the majority of users, reducing the likelihood of submitting broken scripts.

\subsection{Risks}

\subsubsection{Browser Cross-Compatibility}

It is rare for websites and web applications to function identically or appear the same across multiple web browsers, due to the varying compatibilities of even the most popular browsers - this can be problematic for many modern websites. We mostly spent time testing AI Racing Market on Google Chrome, yet tested on Mozilla Firefox and Safari occasionally. Needless to say, this does not replace proper cross-browser compatibility testing using the many tools (such as emulation environments) available to developers these days. However we did not deem this a significant risk, as G-Research employees should be able to iron out compatibility issues prior to a careers fair, installing and choosing the most functional browser in advance. With most of the well known browsers, it is likely that the user interface will simply appear jumbled yet remain functional.

\subsubsection{Mobile Compatibility}

Although users will not be able to run the game itself from a mobile device, it may be possible for students to submit their AI scripts from mobile devices at a careers fair (assuming G-Research permits it, given they will be hosting a local network). This could reduce queues for workstations at the stand. Unfortunately we did not develop our website with mobile users on mind, and some website components are coded such that they would perform poorly on small-resolution devices. This means there is a considerable risk of poor compatibility on mobile devices, that could greatly detract from a mobile user's experience. For example, the Ace editors do not resize gracefully, and as a result the user may struggle to view their scripts with full line width.

\subsection{Challenges}

\subsubsection{Streamlining User Interactions}

To prevent queues to play AI Racing Market at the G-Research stand, we focused on minimizing the user interactions needed to reach the script submission screen. As we were inexperienced web developers at the start of the project, we relied heavily on Bootstrap to provide web components, hence were limited in how unique the user experience could be. Compared to any bespoke solution, using an off-the-shelf alternative (Bootstrap) posed the challenge of ensuring we could design an intuitive website. We describe our design philosophy and how we tackled this challenge in the User Interface Design section, later in this chapter.

\subsubsection{Unity Web-player Loading Times}

The majority of our website was lightweight, with simple components and little dynamic content. However, the Unity web player (which runs and displays the racing), loaded slowly and can occasionally freeze somebody's browser. This is largely out of our control, and we faced the challenge of ensuring a smooth user experience. When running a multiplayer race or a tournament, the user will be waiting for the race to start, and simply waiting for the Unity web-player's loading bar to fill. However, on the edit screen (described later in the User Interface Design section), the user is presented with not only the web-player, but an Ace editor, so that they can make changes to their script and get immediate feedback. 

It was clear to us that we had to avoid re-launching the Unity web-player when the user made changes to their script, as this could potentially freeze and crash someone's browser before they are able to save their updated script. This meant we had to run the Unity web-player in the background, or alongside the editor. We chose to run the web-player underneath the editor for two main reasons: computers with a large screen resolution will be able to view both their script and the editor, creating a more responsive editing experience; and the obvious approach to hiding the Unity web-player without restarting it was minimizing it to a 1x1 pixel resolution, however this resulted in Unity engine errors when initialising the web-player, making the approach infeasible. 

\section{Unity Application}
The Unity application is given the script names for each car and the track to use from the browser as well as the game mode of the race which determines what the application will do at the end of the race:
\begin{itemize}
  \item \textbf{Test Mode} used when editing a script. Creates an endless race which may be restarted by the user at any point.
  \item \textbf{Multiplayer Mode} used when manually setting up a race. The race will end normally and update the leaderboard. The user will be redirected to the leaderboard page at the end of the race.
  \item \textbf{Tournament Mode} only used during tournament races, this mode is similar to the multiplayer mode, but the user will be redirected to the next race instead of the leaderboard.
\end{itemize}

\subsection{API Design}
The key component of the Unity application is the scripting API and our first step in its design was to decide how to give information about the race track so the user can complete a lap. We eventually settled on making this very simple for the user - we drew a line along the centre of the track and let the user set the speed of the car. Our API will automatically steer the car along the centre line at the set speed. Our reasoning on making steering so simple was that guiding the car around the track is the first and least interesting step and we did not want every user to spend the first few minutes writing the same line following script. However this caused races to disintegrate into a train of cars following the same line around the track where the car in first place at the start of the race will be the winner. We solved this creating two more lines for the cars to follow on either side of the centre line and providing API calls to switch to the line on the left or on the right. These calls can be used to overtake cars or to switch to the inside line of a corner.

Finally, we added a few more features which scripts can use to distinguish themselves: the distance to the next corner, the curvature of that corner and a ``boost'' ability which gives the car a large speed increase but may only be used every few seconds. A good script can make use of these features by boosting along straight sections and slowing down according to the sharpness of the corner.

\subsection{Risks}

\subsubsection{Experience with Unity}

Most of our team was inexperienced in the Unity game engine; this posed a
potential issue since it could hinder our development time while everyone learns
the engine's workflow and quirks. However, a good part of these projects is
learning new tools and technologies and if we had all used it before then we
wouldn't really be learning anything new. Thankfully, one of our team members
had previously used Unity and so could quickly explain how things work and the
best practices to follow.

\subsubsection{Artwork and Models}
As computer scientists, creating new models isn't really our forte. None of us
had any side experience with art either, yet this project required a 3D
graphical racing simulation to be shown to the user. Whilst we could probably
have developed some assets ourselves if absolutely necessary, upon investigating
the Unity assets we found them to be sufficient for our uses, and there were
also extra ones available from Unity's Asset Store, which were licensed
appropriately for G-Research's needs.

\subsubsection{API Depth}
The aim of the scripting API is to be complex enough to allow for a wide range of scripts and give users good control over the car, but simple enough to learn and use within 5-10 minutes, so we faced the risk of creating an API which did not find a balance between the two. In an early meeting with Ed, we decided we could mitigate this risk by designing API functions for both extremes where the simple functions alone should be enough to create a good script and users with more time could extend these scripts by using the complex function set. The simple function eventually included switch lane and set speed, while the complex functions included boost and distance to next corner. We also included the event based API to help users with very little programming experience as these scripts will resemble instructions in English.

\subsection{Challenges}

\subsubsection{Steering the Car}
To guide the car around the track we initially placed invisible walls along the edge of the track and created proximity sensors on either side of the car to measure the distance to these walls. The output of the sensors was used to steer the car toward the current lane. The main problem with this approach was that the cars would always drive forward even when pointing the wrong way after skidding or a crash causing the cars to drive the wrong way around the track.

\begin{figure}[H]
\centering
\includegraphics[width=0.75\textwidth]{spline.png}
\caption{The centre spline as seen in the Unity editor window.}
\end{figure}

We solved this problem by replacing the walls and proximity sensors with a spline along the centre of the track and keeping a target position on this spline. The car will always steer toward this target position and the target will always be a small distance in front of the car. This ensures that the car will always drive the right way along the track. Another advantage of the splines is that if two cars take a corner too quickly, the car moving faster will end up further from the track and therefore lose more time than the slower car. With the old approach, both cars would simply bounce off the walls and the faster car would not be penalised more than the slower car.

\subsubsection{Tournament Closing Conditions \& Ranking}
% Define "ranking of the cars" for the reader?
Users are able to submit scripts which cannot finish a race in a reasonable time or which will never finish e.g. cars which flip over in corners, cars which only drive backwards or stay at the start line and cars which override the default steering and go off track. So our tournament must be able to terminate a race before all cars have finished if necessary and still provide a good ranking of the cars.

\begin{figure}[H]
\centering
\includegraphics[width=0.75\textwidth]{HUD.png}
\caption{Cars which have completed the race are shown in grey, cars which have timed out are in red.}
\end{figure}

We accomplished this by placing invisible checkpoints to split each track into short straight sections. We keep maintain the next checkpoint and the time since the last checkpoint was completed for each car. From this we can tell if a car has finished (when the last checkpoint in the race is completed for the third time) or if the car has timed out (if the last checkpoint was completed more than 30 seconds ago). The 30 second time limit is very generous and even a cautious script can easily complete each checkpoint in this time, so a car which times out will belong to an incorrect script. The tournament is ended when all cars have either completed the race or have timed out. Cars which completed the race will be ranked by their time, while cars which have timed out will be ranked according to the number of checkpoints completed, if this is equal, the straight line distance to the next checkpoint is used instead.

\section{User Interface Design}

\subsection{Design Philosophy}

At a a busy careers fair we expect students will have roughly 5-10 minutes sessions with AI Racing Market. Ideally students will have the chance to return and improve their scripts, yet this may be infeasible if there are a lot of interested students. Given students will be short on time, under pressure and likely new to the game, we focused on :
\vspace{-1mm}
\begin{enumerate} \itemsep -2pt 
\item Simplicity
\item Minimal Interactions
\end{enumerate}

We previously justified our use of Bootstrap by the provision of a familiar user experience. By providing the user with a simplistic and familiar interface, users need little explanation of website navigation and can simply focus on script editing and submission. 

An example of how we achieved this was a common navigation bar seen on each web page. This functions as a website backbone, providing the user with access to all of the website's  features. Our template engine, Jade, made this incredibly simple (as shown below); we wrote a basic navigation template that extended our generic Bootstrap layout, which could then be included within the other templates (in Node.js, these are called {\it views}). 

\begin{figure}[H]
\centering
\includegraphics[width=\textwidth]{jadenavigator.png}
\caption{Jade navigator template included within each view.}
\end{figure}

\subsection{Anonymous Script Submission}

Few things kill the excitement of a new game like a lengthy registration process, especially when handing over your email could spell nothing but more email spam. Although we trust that G-Research wont spam students using AI Racing Market, or sell their email addresses to third parties, we designed the script submission such that the users did not have to register. Not only does reduce the interactions needed for a new user to get involved with racing, but it allows the users to remain anonymous, removing the concern that a poor performing script tied to their email address will result in G-Research immediately dismissing their application. As shown below, the user simply associates their work with a script name, and can immediately create a script.

\begin{figure}[H]
\centering
\includegraphics[width=0.8\textwidth]{anonymoussubmit.png}
\caption{Admins will see additional features on their navigation bar after logging in.}
\end{figure}

As we discuss later in this section, G-Research wants to harvest user data.  Therefore we added an ``Optional Details" panel on the right hand side of the script submission screen, allowing users to provide G-Research with information if they wish. If these details are submitted, they are not tied to an account or script until a user registers with the same email address. This means G-Research can recognize that a student has attended their careers fair stand, however their performance can't be scrutinised. If a user is already logged in and wants to submit a new script, they will face the same screen as shown above, excluding the optional details pane.

\subsection{Responsive Script Editing}

Clearly the script submission screen presented above is limited; it provides basic syntax checking, however users will have no feedback on how their scripts perform prior to racing in the tournament. If a user has registered, they can navigate to a more responsive editing mode by clicking on of their submitted scripts, listed on their profile (shown below).

\begin{figure}[H]
\centering
\includegraphics[scale=0.35]{profiletoedit.png}
\caption{Registered users have their scripts listed on their profile, these can be edited on click.}
\end{figure}

After selecting a script to edit, they are presented with not only an Ace editor, but a test race (shown below). Although the user cannot choose the car and track they test their scripts with, they can practice against an opponent script of their choice (without altering the leaderboard rankings). This allows for a more efficient and iterative development process, as the immediate responsive feedback to changes allows the user to incrementally improve their script. To improve usability, the script editing screen is incredibly similar to the script submission screen. 

\begin{figure}[h]
\makebox[\textwidth]
{
	\includegraphics[width=0.49\textwidth]{editscreen.png}
	\hfill    
	\includegraphics[width=0.49\textwidth]{editscreenhide.png}
}
\caption{Edit Screen - users can edit their scripts and test any changes made.}
\end{figure}

There are a few key differences between the script submission and editing screen, beyond the rather obvious Unity Web Player :

\begin{enumerate}
\item Hide button - Now that the user can also view their script racing alongside the Ace editor, we provided a hide button which allows the user to collapse the editing pane, displaying the race without forcing the user to scroll down the page. We decided not to display the race alongside the editor on the screen, as this either requires an unrealistically wide screen resolution, or would reduce the size of the editor and Unity web player drastically. We tested the edit screen using tabs, with the editing pane on one and the race on the other, however we experienced issues resizing the  Unity Web Player and running it in the background.
\item Restore button - After the editing pane has been collapsed, a large restore button is presented where the pane previously was. This is large to ensure that a new user does not accidentally hide the edit screen, struggle to restore it, and consequently leave the page without submitting changes. We could have used a more subtle restore button, and a  pop-up to notify the user of how to find it, but we did not want to obscure the Unity Web Player.
\item Update/Submit/Reset buttons - These new buttons are clustered together in the bottom right of the editing pane, below the Ace editor, to ensure they are visible when the computer being used has a small monitor resolution or if the web-browser has a high zoom level. Although clustering these buttons risks users accidentally submitting changes, we ensured the buttons had contrasting colours to avoid confusion.
	\begin{enumerate}
	\item Submit button - As the user is logged in, submitting changes updates the user's script and returns them to their profile. They are greeted by a pop-up that notifies them that their changes had been made successfully.
	\item Update button - Pressing the update button re-interprets the script, whether changes have been made or not. This allows the user to observe changes on demand, without restarting the race. 
	\item Reset button - Pressing the reset button restarts the race, re-interpreting the script and changing the opponent to the one selected in the ``Enemy" drop-down menu.
	\end{enumerate}
\end{enumerate}

\subsection{Admin Features}

We restricted a few features from the average user, which are only accessible by a registered admin, i.e. G-Research employee. These features were the tournament mode and admin panel, as shown below.

\begin{figure}[H]
\centering
\includegraphics[width=\textwidth]{admintoolbar.png}
\caption{Admins will see additional features on their navigation bar after logging in.}
\end{figure}

These REST end points {\tt/tournament} and {\tt/admin} aren't secured from other users or anonymous users, however we do not expect users to access these features. This is mostly because users will be supervised by G-Research recruiters, but also because users will simply not know the names of these routes, and could only access them by trial and error.

The tournament button on the navigation bar will display a tournament, with races being run automatically with the leaderboard changes being displayed after each race. The admin button displays the personal details of registered users and the names of a user's scripts (shown below) - this feature was requested by our client. 
\begin{figure}[H]
\centering
\includegraphics[width=\textwidth]{adminpanel.png}
\caption{Admins can view the details of registered users.}
\end{figure}

\section{Script Execution}
The architecture of our first prototype differed significantly from that of the final product. Originally, we required users to write their AI scripts in Unity Script - a bastardisation of JavaScript. These scripts were directly inserted into the Unity engine and executed on the car objects. While this design proved easy to implement, it offered some serious flaws:
\begin{enumerate}
\item Unity Script, while syntactically similar to JavaScript, has several key differences. For example, Unity Script does not offer the dynamic objects that are a core component of JavaScript. These differences would need to be taught to users.
\item Passing state between sequential executions of the script (such as, has a car recently been overtaken?) had to be done with a predefined global hash table. Users were unable to add their own variables to global scope.
\item Executing directly on a game object offers serious security concerns. Users with basic knowledge of unity could bypass our defined API and access the car's private data. Particularly cunning users could modify the car's x and y coordinates, teleporting the car to the finish line!
\end{enumerate}

\begin{figure}[H]
\centering
\begin{subfigure}{.5\textwidth}
  \centering
\begin{lstlisting}[language=JavaScript]  
global["speed"] = 20;

if(farFromCorner) {
  global["speed"] 
    = global["speed"] + 1;
}

\end{lstlisting}
  \caption{Unity Script}
  \label{fig:sub1}
\end{subfigure}%
\begin{subfigure}{.5\textwidth}
  \centering
\begin{lstlisting}[language=JavaScript]  
var speed = 20;

if(farFromCorner) {
  speed++;
}
\end{lstlisting}
  \caption{JavaScript}
  \label{fig:sub2}
\end{subfigure}
\caption{Comparison of prototype AI script and modern AI script.}
\label{fig:unityvsjava}
\end{figure}

To solve these issues we redesigned our front-end architecture by uncoupling the game object and AI script - replacing the direct calls with proxies. Now the AI script remains in the browser, never seen by unity, where it receives a serialised view of the game state and outputs a list of moves. This abstraction immediately solved (3) since the the proxy prevents any potentially dangerous requests like teleportation. 

Once a proxy had been defined the AI script could be in any language we choose, provided it handles the requests dispatched by unity. As the most popular web language, JavaScript is a natural choice for this - solving (1). Finally, now that the script is run in a JavaScript context, global variables can be easily stored in the browser window. Thus, the only problem remaining to solve (2) is isolating scripts global state (for example two scripts may both have a global "speed" variable, which should be sperate). We achieved this by binding each script inside a anonymous lambda, faking global state by using function local state.
